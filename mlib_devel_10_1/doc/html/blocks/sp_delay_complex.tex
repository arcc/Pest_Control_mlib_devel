\documentclass{article}
\oddsidemargin  0.0in
\evensidemargin 0.0in
\textwidth      6.5in
\usepackage{tabularx}
\usepackage{html}
\title{\textbf{CASPER Library} \\Reference Manual}
\newcommand{\Block}[6]{\section {#1 \emph{(#2)}} \label{#3} \textbf{Block Author}: #4 \\ \textbf{Document Author}: #5 \subsection*{Summary}#6}

\newenvironment{PortTable}{\subsection*{Ports}
\tabularx{6.5in}{|l|l|l|X|} \hline  \textbf{Port} & \textbf{Dir.} & \textbf{Data Type} & \textbf{Description} \\ \hline}{\endtabularx}

\newcommand{\Port}[4]{\emph{#1} & \lowercase{#2} & #3 & #4\\  \hline}

\newcommand{\BlockDesc}[1]{\subsection*{Description}#1}

\newenvironment{ParameterTable}{\subsection*{Mask Parameters}
\tabularx{6.5in}{|l|l|X|} \hline  \textbf{Parameter} & \textbf{Variable} & \textbf{Description} \\ \hline}{\endtabularx}

\newcommand{\Parameter}[3]{#1 & \emph{#2} & #3 \\ \hline}

\begin{htmlonly}
\newcommand{\tabularx}[3]{\begin{tabularx}{#1}{#2}{#3}}
\newcommand{\endtabularx}{\end{tabularx}}
\end{htmlonly}

\date{Last Updated \today}
\begin{document}
\maketitle

%\chapter{System Blocks}
%%%%Change Chapter%%%%%%%%
%\chapter{Signal Processing Blocks}

%\input{test.tex}
%\chapter{Communication Blocks}
%\end{document} 
\Block{The Complex Delay Block}
{delay\_complex}
{delaycomplex}
{Aaron Parsons}
{Aaron Parsons}
{A delay block that treats its input as complex, splits it into real and imaginary components, delays each component by a specified amount, and then re-joins them into a complex output. The underlying storage is user-selectable (either BRAM or SLR16 elements). The reason for this is wide (36 bit) delays make adjacent multipliers in multiplier-bram pairs unusable.}



\begin{ParameterTable}

\Parameter{Delay Depth}{delay\_depth}{The length of the delay.}

\Parameter{Bit Width}{n\_bits}{Specifies the width of the real/imaginary components. Width of each component is assumed equal.}

\Parameter{Use BRAM}{use\_bram}{Set to 1 to implement the delay using BRAM. If 0, the delay will be implemented using SLR16 elements.}

\end{ParameterTable}



\begin{PortTable}

\Port{in}{in}{???}{The complex signal to be delayed.}

\Port{out}{out}{???}{The delayed complex signal.}

\end{PortTable}



\BlockDesc{A delay block that treats its input as complex, splits it into real and imaginary components, delays each component by a specified amount, and then re-joins them into a complex output. The underlying storage is user-selectable (either BRAM or SLR16 elements). The reason for this is wide (36 bit) delays make adjacent multipliers in multiplier-bram pairs unusable.} 
\end{document}

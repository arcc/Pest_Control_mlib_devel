\documentclass{article}
\oddsidemargin  0.0in
\evensidemargin 0.0in
\textwidth      6.5in
\usepackage{tabularx}
\usepackage{html}
\title{\textbf{CASPER Library} \\Reference Manual}
\newcommand{\Block}[6]{\section {#1 \emph{(#2)}} \label{#3} \textbf{Block Author}: #4 \\ \textbf{Document Author}: #5 \subsection*{Summary}#6}

\newenvironment{PortTable}{\subsection*{Ports}
\tabularx{6.5in}{|l|l|l|X|} \hline  \textbf{Port} & \textbf{Dir.} & \textbf{Data Type} & \textbf{Description} \\ \hline}{\endtabularx}

\newcommand{\Port}[4]{\emph{#1} & \lowercase{#2} & #3 & #4\\  \hline}

\newcommand{\BlockDesc}[1]{\subsection*{Description}#1}

\newenvironment{ParameterTable}{\subsection*{Mask Parameters}
\tabularx{6.5in}{|l|l|X|} \hline  \textbf{Parameter} & \textbf{Variable} & \textbf{Description} \\ \hline}{\endtabularx}

\newcommand{\Parameter}[3]{#1 & \emph{#2} & #3 \\ \hline}

\begin{htmlonly}
\newcommand{\tabularx}[3]{\begin{tabularx}{#1}{#2}{#3}}
\newcommand{\endtabularx}{\end{tabularx}}
\end{htmlonly}

\date{Last Updated \today}
\begin{document}
\maketitle

%\chapter{System Blocks}
%%%%Change Chapter%%%%%%%%
%\chapter{Signal Processing Blocks}

%\input{test.tex}
%\chapter{Communication Blocks}
%\end{document} 
\Block{The Partial Delay Block}{partial\_delay}{partialdelay}{Aaron Parsons}{Aaron Parsons}{For a set of parallel inputs which represent consecutive time samples of the same input signal, this block delays the stream by a dynamically selectable number of samples between 0 and (n\_inputs-1).}



\begin{ParameterTable}

\Parameter{No. of inputs.}{n\_inputs}{The number of parallel inputs.}

\Parameter{Mux Latency}{latency}{The latency of each mux block.}

\end{ParameterTable}



\begin{PortTable}

\Port{sync}{???}{???}{Indicates the next clock cycle containing valid data}

\Port{din}{in}{???}{A number to be summed.}

\end{PortTable}



\BlockDesc{For a set of parallel inputs which represent consecutive time samples of the same input signal, this block delays the stream by a dynamically selectable number of samples between 0 and (n\_inputs-1). This is useful for blocks such as the ADC that present several samples in parallel because sampling occurs at a higher clock rate than that of the FPGA.





\begin{table*}[ht]

	\centering

		\begin{tabular}{|c|c|c|c||c||c|c|c|c|}

		\hline

			...	&	4	&	0	&	...	&	$\rightarrow$	&	6	&	2	&	...	&	...	\\ \hline

	 		...	&	5	&	1	&	...	&	$\rightarrow$	&	7	&	3	&	...	&	...	\\ \hline

			...	&	6	&	2	&	...	&	$\rightarrow$	&	...	&	4	&	0	&	...	\\ \hline

			...	&	7	&	3	&	...	&	$\rightarrow$	&	...	&	5	&	1	&	...	\\ \hline

		\end{tabular}	\caption{Mapping of 4 parallel input samples to output for delay = 2}

	\label{tab:MappingOf4ParallelInputSamplesToOutputForDelay2}

\end{table*}



} 
\end{document}

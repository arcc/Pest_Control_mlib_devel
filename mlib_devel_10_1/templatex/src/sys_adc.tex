%%%%%%%%%%%%%%%%%%%%%%%%%%%%%%%%%
%%%Template for documenting CASPER Library Blocks%%%
%%%%%%%%%%%%%%%%%%%%%%%%%%%%%%%%%

%%%\Block describes a Block. The simulink name should be exactly as it appears in 
%%%Simulink, note that \_ must be used for underscores.
%%%The block label is the same as the simulink name but without the underscores.
%%%This label is defined as a latex reference and is
%%%usefull for cross-referencing other blocks. You can include in your
%%%Block Descriptions commands like refer to the adder tree block \ref{addertree}.
%%%The Block description can include arbitrary Latex, but do not use the Latex Math 
%%%%environment, it causes the conversion to HTML complicated. 
%%%Use \emph{} to write math and try to use MATLAB commands for math expressions
%%%All Parameters for the \Block \Port and \Parameter commands must be present.

\Block{ADC}{adc}{adc}{Pierre Yves Droz}{Ben Blackman}{The ADC block converts analog inputs to digital outputs. Every clock cycle, the inputs are sampled and digitized to 8 bit binary point numbers in the range of [-1, 1) and are then output by the adc.}

\begin{ParameterTable}
\Parameter{ADC board}{adc\_brd}{Select which ADC port to use on the IBOB.}
\Parameter{ADC clock rate (MHz)}{adc\_clk\_rate}{Sets the clock rate of the ADC, must be at least 4x the IBOB clock rate.}
\Parameter{ADC interleave mode}{adc\_interleave}{Check for 1 input, uncheck for 2 inputs.}
\Parameter{Sample period}{sample\_period}{Sets the period at which the adc outputs samples (ie 2 means every other cycle).}
\end{ParameterTable}

\begin{PortTable}
\Port{sim\_in}{IN}{double}{The analog signal to be digitized if interleave mode is selected. Note: For simulation only.}
\Port{sim\_i}{IN}{double}{The first analog signal to be digitized if interleave mode is unselected. Note: For simulation only.}
\Port{sim\_q}{IN}{double}{The second analog signal to be digitized if interleave mode is unselected. Note: For simulation only.}
\Port{sim\_sync}{IN}{double}{Takes a pulse to be observed at the output to measure the delay through the block. Note: For simulation only.}
\Port{sim\_data\_valid}{IN}{double}{A signal that is high when inputs are valid. Note: For simulation only.}
\Port{oX}{OUT}{Fix\_8\_7}{A signal that represents sample X+1 (Ex. o0 is the 1st sample, o7 is the 8th sample). Used if interleave mode is on.}
\Port{iX}{OUT}{Fix\_8\_7}{A signal that represents sample X+1 (Ex. i0 is the 1st sample, o3 is the 4th sample). Used if interleave mode is off.}
\Port{qX}{OUT}{Fix\_8\_7}{A signal that represents sample X+1 (Ex. q0 is the 1st sample, q3 is the 4th sample). Used if interleave mode is off.}
\Port{outofrangeX}{OUT}{boolean}{A signal that represents when samples are outside the valid range.}
\Port{syncX}{OUT}{boolean}{A signal that is high when the sync pulse offset by X if interleave mode is unselected, or 2X if interleave mode is selected is high (Ex. sync2 is the pulse offset by 2 if interleave is off or offset by 4 if interlave is on).}
\Port{data\_valid}{OUT}{boolean}{A signal that is high when the outputs are valid.}

\end{PortTable}

\BlockDesc{
\paragraph{Usage}
The ADC block can take 1 or 2 analog input streams. The first input should be connected to input i and the second to input q if it is being used. The inputs will then be digitized to \textit{Fix\_8\_7} numbers between [-1, 1). For a single input, the \textit{adc} samples its input 8 times per IBOB clock cycle and outputs the 8 samples in parallel with o0 being the first sample and o7 the last sample. For 2 inputs, the \textit{adc} samples both inputs 4 times per IBOB clock cycle and then outputs them in parallel with i0-i3 corresponding to input i and q0-q3 corresponding to input q. In addition to having 2 possible inputs, each IBOB can interface with 2 \textit{adc}s for a total of 4 inputs or 2 8-sample inputs per IBOB. 

\paragraph{Connecting the Hardware}
To hook up the ADC board, attach the clock SMA cable to the clk\_i port, the first input to the I+ port, and the second input to the Q+ port. Check the hardware on the ADC board near the input pins. There should be for 4 square chips in a straight line. If there are only 3, the second input, Q+, may not work. Note that if you chose \textit{adc0\_clk}, make sure to plug the ADC board in to the adc0 port. The same applies if you chose \textit{adc1\_clk} to plug the board into adc1 port. If you are using both ADCs, then you need to plug a clock into both clk\_i inputs and you should probably run them off of the same signal generator. 

\paragraph{ADC Background Information}
The ADC board was designed to mate directly to an IBOB board through ZDOK connectors for high-speed serial data I/O. Analog data is digitized using an Atmel AT84AD001B dual 8-bit ADC chip which can digitize two streams at 1 Gsample/sec or a single stream at 2 Gsample/sec. This board may be driven with either single-ended or differential inputs. 

}

